% Template created by Karol Kozioł (www.karol-koziol.net) for ShareLaTeX

\documentclass[a4paper,spanish,9pt]{extarticle}
\usepackage[utf8]{inputenc}
\usepackage[T1]{fontenc}
\usepackage{graphicx}
\usepackage{xcolor}
\usepackage{tikz}
\usetikzlibrary{shapes, calc, shapes, arrows, babel}

\usepackage{amsmath,amssymb,textcomp}
\everymath{\displaystyle}

\usepackage{times}
\renewcommand\familydefault{\sfdefault}
\usepackage{tgheros}
\usepackage[defaultmono,scale=0.85]{droidmono}

\usepackage{multicol}
\setlength{\columnseprule}{0pt}
\setlength{\columnsep}{20.0pt}

\usepackage[utf8]{inputenc}
\usepackage[spanish]{babel}
\usepackage{eurosym}

\usepackage{graphicx}
\graphicspath{{../img/}}
\usepackage{svg}

\usepackage{hyperref}

\usepackage{geometry}
\geometry{
a4paper,
total={210mm,297mm},
left=10mm,right=10mm,top=10mm,bottom=15mm}

\linespread{1.3}


% custom title
\makeatletter
\renewcommand*{\maketitle}{%
\noindent
\begin{minipage}{0.6\textwidth}
\begin{tikzpicture}
\node[rectangle,rounded corners=6pt,inner sep=10pt,fill=blue!50!black,text width= 0.95\textwidth] {\color{white}\Huge \@title};
\end{tikzpicture}
\end{minipage}
\hfill
\begin{minipage}{0.35\textwidth}
\begin{tikzpicture}
\node[rectangle,rounded corners=3pt,inner sep=10pt,draw=blue!50!black,text width= 0.95\textwidth] {\begin{tabular}{cc} \multirow{2}{1cm}{\includegraphics[width=0.15\columnwidth]{header_right}}& \@author \\ & \ies \end{tabular}};
\end{tikzpicture}
\end{minipage}
\bigskip\bigskip
}%
\makeatother

% custom section
\usepackage[explicit]{titlesec}
\newcommand*\sectionlabel{}
\titleformat{\section}
  {\gdef\sectionlabel{}
   \normalfont\sffamily\Large\bfseries\scshape}
  {\gdef\sectionlabel{\thesection\ }}{0pt}
  {
\noindent
\begin{tikzpicture}
\node[rectangle,rounded corners=3pt,inner sep=4pt,fill=blue!50!black,text width= 0.95\columnwidth] {\color{white}\sectionlabel#1};
\end{tikzpicture}
  }
\titlespacing*{\section}{0pt}{15pt}{10pt}


% custom footer
\usepackage{fancyhdr}
\makeatletter
\pagestyle{fancy}
\fancyhead{}
\fancyfoot[C]{\footnotesize \@author \ - \ies}
\renewcommand{\headrulewidth}{0pt}
\renewcommand{\footrulewidth}{0pt}
\makeatother
\usepackage{multirow} % para las tablas


\title{Trigonometría}
\author{Departamento de Matemáticas}
\date{2014}
\newcommand{\ies}{IES Pedro Cerrada}



\begin{document}

\maketitle

 \begin{tikzpicture}[scale=3]
 \draw[step=.5cm, gray, very thin] (-1.2,-1.2) grid (1.2,1.2); 
 \filldraw[fill=green!20,draw=green!50!black] (0,0) -- (3mm,0mm) arc (0:30:3mm) -- cycle; 
 \draw[->] (-1.25,0) -- (1.25,0) coordinate (x axis);
 \draw[->] (0,-1.25) -- (0,1.25) coordinate (y axis);
 \draw (0,0) circle (1);
 \draw[very thick,red] (30:1) -- node[left,fill=white] {$\sen \alpha$} (30:1 |- x axis);
 \draw[very thick,blue] (30:1cm |- x axis) -- node[below=2pt,fill=white] {$\cos \alpha$} (0,0);
 \draw (0,0) -- (30:1cm);
 \foreach \x/\xtext in {-1, -0.5/-\frac{1}{2}, 1} 
   \draw (\x cm,1pt) -- (\x cm,-1pt) node[anchor=north,fill=white] {$\xtext$};
 \foreach \y/\ytext in {-1, -0.5/-\frac{1}{2}, 0.5/\frac{1}{2}, 1} 
   \draw (1pt,\y cm) -- (-1pt,\y cm) node[anchor=east,fill=white] {$\ytext$};
 \end{tikzpicture}


\begin{multicols*}{2}


\section{Potencias}

Es importante destacar que las propiedades se pueden leer (y por tanto aplicar) de izquierda a derecha o al revés.
$$\forall \ n,m \in \mathbb{N} \ y \ \forall \  n,m \in \mathbb{R}:$$
\begin{tabular}{ll}
\textbf{Definición} de potencia: & $a^n = a \cdot a \stackrel{n}{\cdots} a$ \\
\textbf{Potencia} de exponente \textbf{negativo}: & $a^{-n} = \dfrac{1}{a^n}$ \\
\textbf{Potencia} de exponente \textbf{0} $\left( Si \ a \neq 0\right)$: & $a^{0} = 1$ \\
\textbf{Producto} de potenc. de la \textbf{misma base}: & $a^n a^m = a^{n+m}$ \\
\textbf{Cociente} de potenc. de la \textbf{misma base}: & $\dfrac{a^n}{a^m}  = a^{n-m}$ \\
\textbf{Potencia} de una \textbf{potencia}: & $\left(a^n\right)^m  = a^{n \cdot m}$ \\
\textbf{Potencia} de un \textbf{producto}: & $\left(a \cdot b \right)^n = a^n \cdot b^n$ \\
\textbf{Potencia} de un \textbf{cociente}: & $\left(\dfrac{a}{b} \right)^n = \dfrac{a^n}{a^n}$
\end{tabular}

\subsection{Ejemplos}

\begin{tabular}{lll}
$2^3=2\cdot2\cdot2$ & $3^0=1$ & $2^{-3}=\dfrac{1}{2^3}$\\
$2^3 \cdot 2^4 = 2^{4+3} = 2^{7}$ & $\left(\dfrac{2}{5}\right)^{-3}=\left(\dfrac{5}{2}\right)^{3}$ & $\dfrac{2^4}{2^3} = 2^{4-3} = 2$\\
$2^5:2^3 = 2^{5-3} = 2^2$ & $\left(3^2\right)^4  = 3^{2 \cdot 4}=3^{8}$ & $\left(\dfrac{1}{2} \right)^3 = \dfrac{1^3}{2^3}$\\
$2^3 \cdot 3^3=\left(2 \cdot 3 \right)^3 = 6^3$
\end{tabular}

\section{Radicales}
Recuerda que: $\sqrt[n]{a}=b \longleftrightarrow b^n=a$. De la definición se deducen las siguientes propiedades:
\vspace{0.3cm}

\begin{tabular}{ll}
\textbf{Forma Exponencial}: & $\sqrt[n]{a}=a^\frac{1}{n} \ \sqrt[n]{a^m}=a^\frac{m}{n} $ \\
\textbf{Simplificación}: & $\sqrt[np]{a^p}=\sqrt[n]{a}$ \\
\textbf{Raíz} de un \textbf{producto}: & $\sqrt[n]{a\cdot b}=\sqrt[n]{a}\cdot\sqrt[n]{b}$ \\
\textbf{Raíz} de un \textbf{cociente}: & $\sqrt[n]{\frac{a}{b}}=\frac{\sqrt[n]{a}}{\sqrt[n]{b}}$ \\
\textbf{Potencia} de un \textbf{radical} & $\left(\sqrt[n]{a}\right)^p=\sqrt[n]{a^p}$ \\
\textbf{Raíz} de un \textbf{radical} & $\sqrt[n]{\sqrt[m]{a}}=\sqrt[n\cdot m]{a}$ \\
\end{tabular}

\vspace{0.5cm}
\textbf{Suma y resta de radicales:} Recuerda que solo se pueden sumar o restar expresiones con radicales idénticos

\vspace{0.5cm}
\textbf{Racionalizar radicales:} Se multiplica el numerador y denominador por un expresión que permita que desaparezcan los radicales del denominador

\subsection{Ejemplos}
\begin{tabular}{lr}
$\sqrt[2]{4}=4^\frac{1}{2}$ & $\sqrt[6]{3^2}=3^\frac{3}{6}=3^\frac{1}{2}$ \\
$\sqrt[4]{5^2}=\sqrt{5}$ & $\sqrt[3]{2^6}=2^2=4$\\
$\sqrt{2\cdot 3}=\sqrt{2}\cdot\sqrt{3}$ & $\sqrt{12}=\sqrt{2^2\cdot 3}=\sqrt{2^2}\cdot\sqrt{3}=2\sqrt{3}$\\ 
$\sqrt[3]{9}\cdot\sqrt[3]{3}=\sqrt[3]{27}=3$ & $\frac{\sqrt{300}}{\sqrt{3}}=\sqrt{\frac{300}{3}}=\sqrt{100}=10$
\end{tabular}

\begin{tabular}{lr}
$\left(\sqrt{2}\right)^2=\sqrt{2^2}=2$ & $\sqrt[3]{\sqrt[5]{3}}=\sqrt[15]{3}$\\
$3\sqrt[3]{7}-\sqrt[3]{7}=2\sqrt[3]{7}$ & $2\sqrt[3]{5}-\sqrt[4]{5}+2\sqrt[3]{5}=4\sqrt[3]{5}-\sqrt[4]{5}$
\end{tabular}

\begin{tabular}{l}
$3\sqrt{28}-\sqrt{7}-\sqrt{63}=3\cdot 2\sqrt{7}-\sqrt{7}-3\sqrt{7}=2\sqrt{7}$ \\
$\frac{1}{\sqrt[6]{5^2}}=\frac{1\cdot \sqrt[6]{5^4}}{\sqrt[6]{5^2}\cdot \sqrt[6]{5^4}}=\frac{\sqrt[6]{5^4}}{5}=\frac{\sqrt[3]{5^2}}{5}$  \\
$\frac{1}{\sqrt{3}-1}=\frac{1 \cdot \left(\sqrt{3}+1\right)}{\left(\sqrt{3}-1 \cdot \right)\cdot \left(\sqrt{3}+1\cdot \right)}=\frac{\sqrt{3}+1}{3-1}=\frac{\sqrt{3}+1}{2}$
\end{tabular}

\section{Logaritmos}

\begin{tabular}{ll}
\textbf{Definición} de logaritmo: & $\log_b x = n \longleftrightarrow b^n=x$\\
Logaritmo de un \textbf{producto}: & $\log_b \left(x\cdot y\right)=log_b x + log_b y$\\
Logaritmo de un \textbf{cociente}: & $\log_b \frac{x}{y}=log_b x - log_b y$\\
Logaritmo de una \textbf{potencia}: & $\log_b {x^n} = n \cdot \log_b x$\\
Logaritmo de una \textbf{raíz}: & $\log_b {\sqrt[n]{x}} = \frac{1}{n} \cdot \log_b x$\\
\textbf{Cambio} de \textbf{base}: & $\log_b x =\frac{\log_a x}{\log_a b}$\\
Logaritmo \textbf{decimal}: & $\log x = log_{10} x$
\end{tabular}\\

\subsection{Ejemplos}

\begin{tabular}{ll}
$\log_3 3 = 1$ & $\log_3 1 = 0$ \\
$\log_2 8 = logb_2 2^3 = 3$  
\end{tabular}

\begin{tabular}{l}
$\log_2 \left(4\cdot 16\right)=\log_2 4 + \log_2 16=2 + 4$  \\
$\log_{10}\frac{1000}{10} =\log_{10} 1000 - \log_{10} 10 = 3-1=2 $ \\
$\log_3 {81^3} = 3 \cdot \log_3 81= 3 \cdot 4= 12$ \\
$\log_3 {\sqrt[4]{81}} = \frac{1}{4} \log_3 81=\frac{1}{4} \log_3 3^4= \frac{1}{4} \cdot 4= 1$ \\
$\log_6 4 =\frac{\log_{10} 6}{\log_{10} 6}$
\end{tabular}



\section{Versión Online}

\url{https://goo.gl/kZNTW4} \includegraphics[width=0.15\columnwidth]{qr_chuletapot}





\end{multicols*}

\end{document}
